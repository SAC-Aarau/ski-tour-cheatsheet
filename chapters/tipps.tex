\section{Diverse Tipps und Tricks}

Der Flüssigkeitsbedarf ist im Winter oft etwas niedriger als im Sommer, hier können wir also Gewicht sparen.

Heisser Tee aus der Thermosflasche kann mit Schnee \enquote{gestreckt} werden, er reicht dann etwas länger und wir sparen Gewicht.

Überflüssiger Proviant ist überflüssiges Gewicht -- nach der Tour sollte nichts mehr übrig sein.

Wenige und lange Kurven sparen Kraft auf der Abfahrt -- im Gegensatz zu vielen kurzen Schwüngen.

Vor dem entfernen der Steigfelle die Bindung wieder auf \enquote{Abfahrt} stellen kann verhindern, dass sich ein Ski selbstständig macht.

Nach der Tour die Felle trocknen: Nicht in der prallen Sonne oder auf dem Heizkörper.

Die Ski nach der Tour voneinander nehmen und einzeln trocknen lassen verhindert rostige Kanten.

Nach der Tour die Innenschuhe herausnehmen und separat trocknen lassen.

Auch das LVS nach der Tour kurz trocknen lassen.

Ein frisch gewachster Belag dreht leichter und fährt sich besser.
Spätestens bei grauem Belag ist die nächste Behandlung mit Heisswachs fällig.

Ein paar Kratzer im Belag sind nicht schlimm, so schmerzt auch der nächste Stein weniger.

\printinunitsof{mm}\prntlen{\linewidth}
\printinunitsof{mm}\prntlen{\textwidth}
\printinunitsof{mm}\prntlen{\columnwidth}
\printinunitsof{mm}\prntlen{\columnsep}