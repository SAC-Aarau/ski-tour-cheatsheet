\section{Über die Ausrüstung}

Um die geeignete Bekleidung für eine Skitour zu finden können wir uns grundsätzlich an den Sommertouren orientieren und eine Schicht hinzufügen.

Wir kleiden uns nach dem Zwiebelprinzip: Viele dünne Schichten sind besser als wenige Dicke.

Eine gute \textit{Tourenbekleidung} schützt vor Wind, Nässe und Kälte -- transportiert aber auch den Schweiss nach aussen.

\begin{itemize}
  \item{Thermo Unterwäsche}
  \item{Skisocken}
  \item{Tourenhosen -- bei nasser Witterung wasserdicht}
  \item{Fleecjacke oder Pullover -- bevorzugt mit Kapuze}
  \item{Hardshelljacke -- Wind- uns Wasserdicht}
  \item{Daunenjacke}
  \item{Buff, Halstuch oder Sturmhaube}
  \item{Kappe -- bevorzugt winddicht}
  \item{Sonnenhut -- bei schönem Wetter}
  \item{dünne und robuste Handschuhe}
  \item{dicke und warme Handschuhe}
  \item{Sonnenbrille -- Schutzklasse 4}
  \item{Tourenskischuhe -- mit Gummisohle}
\end{itemize}

\textit{Die Hosen sollten auch im Aufstiegsmodus über die Skischuhe reichen und trotzdem nicht zu locker sitzen.}

Leichte und aufstiegsorientierte Skischuhe sind zwar angenehm zu tragen, machen sich aber durch ihre fehlende Stabilität meist negativ auf der Abfahrt bemerkbar -- insbesondere bei schwierigen Schneeverhältnissen.

\textit{Der etwas schwerere Schuh mit 3 oder 4 Schnallen ist also dem leichten Modell mit Klettverschluss und 1 oder 2 Schnallen vorzuziehen.}

Ausserdem benötigen wir eine \textit{Tourenausrüstung}:

\begin{itemize}
  \item{Rucksack -- etwa 30 l für eine Wochenendtour}
  \item{Teleskopstöcke -- mit grossen Tellern}
  \item{Tourenski -- mit eingestellter Bindung}
  \item{Steigfelle -- passend zum Ski}
  \item{Harscheisen -- passend zu Ski und Bindung}
  \item{Stirnlampe -- besonders im Hochwinter}
  \item{Sonnencreme -- vor allem im Frühling}
\end{itemize}

\textit{Der Rucksack sollte über eine Befestigungsmöglichkeit für Ski, Stöcke und Pickel verfügen.}

\newcolumn

\textit{Schneesportgeräte für die Piste eignen sich durch ihre Geometrie und Gewicht nur äusserst bedingt für den Toureneinsatz.}

Kurze Ski erleichtern die Spitzkehre im Aufstieg und sind wendiger auf der Abfahrt, allerdings auch etwas nervöser.
Verlangen also nach mehr Kontrolle. Die Skilänge ist abhängig von der Körpergrösse und sollte ab Boden etwa zwischen Kinn und Stirn liegen.

Ein klassischer Tourenski ist unterhalb der Bindung etwa 88 mm breit.
Wer mehr auf Abfahrt setzt greift zum etwas breiteren Ski mit 95 mm oder breiter.

Tourenski unterscheiden sich vom Pistenski neben dem Gewicht vor allem durch die früher auf biegende Skispitze.
Dadurch erhält der Ski im Tiefschnee mehr Auftrieb.
Gleichzeitig verringert sich bei harten Bedingungen und auf der Piste aber die Kontaktfläche zwischen Schneeoberfläche und Ski.
Womit der Ski während der Fahrt nervöser und schwerer zu kontrollieren wird.

Bei der Bindung hat sich das Pin System durchgesetzt.
Diese zeichnet sich durch geringeres Gewicht und besseren Drehpunkt im Aufstieg aus.
Tourenskischuh und Bindung sind in der Regel frei kombinierbar.
Beim Kauf eines Occasion Ski sollte allerdings die Sohlenlänge beachtet werden.

Ohne Rennambitionen sollte darauf geachtet werden, dass bei der Bindung der sogenannte Z-Wert eingestellt werden kann.
Der persönliche Z-Wert sollte sich ausserdem nicht gerade am Anfang oder Ende des einstellbaren Bereichs befinden.
Da wir Abseits eher langsamer unterwegs sind empfiehlt es sich den Z-Wert etwas tiefer als beim Pistenski zu wählen.

Neben geeigneter Bekleidung und Tourenausrüstung benötigen wir eine \textit{Notfallausrüstung}:

\begin{itemize}
  \item{LVS -- mit 3 Antennen und genug Batterieladung}
  \item{Lawinensonde -- aus Alu oder Carbon}
  \item{Lawinenschaufel -- mit einem Metallblatt}
  \item{Lawinenairbag -- kann Leben retten}
  \item{Helm -- ist sehr empfohlen}
\end{itemize}

\textit{Eine gute Schaufel kann im Notfall zwischen Leben und Tod entscheiden. Wir tragen dieses wichtige Ausrüstungsstück nicht für uns mit -- sondern für alle anderen.}

Im weiteren benötigen wir innerhalb der Gruppe:

\begin{itemize}
  \item{Kommunikationsmittel -- Netzabdeckung prüfen}
  \item{Notfallapotheke -- selber Inhalt wie im Sommer}
  \item{Biwaksack}
  \item{Reparaturset}
\end{itemize}

Neben einer Verletzung kann auch unsere Ausrüstung unterwegs kaputt gehen.
Deshalb sollten wir stets auch ein kleines Reparaturset mitführen, um im Notfall improvisieren zu können:

\begin{itemize}
  \item{Kabelbinder -- umso stabiler desto besser}
  \item{Zaundraht -- etwa 1 mm Durchmesser}
  \item{Reepschnur -- etwa 2 bis 5 Meter}
  \item{Taschenmesser -- mit Schraubenzieher und Zange}
  \item{Graphitkerze -- zum Löcher im Skibelag auffüllen}
  \item{Feuerzeug -- für die Graphitkerze}
  \item{Fellwachs -- hilfreich bei Stollenbildung}
\end{itemize}

\newcolumn

Im Skibergsteigen und für Touren mit einer Tragepassage oder Fussaufstieg zum Gipfel ist eine technische Zusatzausrüstung notwendig:

\begin{itemize}
  \item{Steigeisen -- mit Antistoll}
  \item{Pickel -- meist genügt ein leichtes Allzweckmodell}
  \item{Anseilgurt -- anziehen mit Skischuhen beachten}
  \item{Seil -- bei ausgesetzten Stellen und auf Gletscher}
  \item{Material für die Spaltenrettung}
  \item{Prusikschlinge und Abseilgerät}
\end{itemize}

\textit{Leichtsteigeisen aus Aluminium fallen kaum ins Gewicht und leisten im reinen Firn gute Dienste. Bei Felskontakt oder Blankeis geraten sie aber sehr schnell an ihre Grenze.}
