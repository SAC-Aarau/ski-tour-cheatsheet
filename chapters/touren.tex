\section{Tourenplanung}

\subsection{Tourenplanung nach dem 3x3 Raster}

Die Tourenplanung ist bis zum Tourenende ein ständiger Prozess, wobei die möglichen Entscheidungen immer weniger werden.

Wir planen eine Tour nach dem 3x3 Raster.

Dabei betrachten wir die drei \textit{Faktoren}:

\begin{itemize}
  \item{
    \textbf{Verhältnisse} -- Wetter, Bulletin, Tourenberichte
    \begin{itemize}
      \item{\href{https://www.meteoschweiz.admin.ch/}{www.meteoschweiz.admin.ch}, MeteoSwiss App}
      \item{\href{https://www.slf.ch/}{www.slf.ch}, White Risk App}
      \item{\href{https://www.gipfelbuch.ch/}{www.gipfelbuch.ch}, Hüttenwart, Bergführerbüro}
    \end{itemize}
  }
  \item{\textbf{Gelände} -- Route, Karte, Führerliteratur
    \begin{itemize}
      \item{\href{https://s.geo.admin.ch/y34y7btkqsvz}{www.map.geo.admin.ch}, White Risk App}
      \item{\href{https://www.skitourenguru.com//}{www.skitourenguru.com}}
      \item{\href{https://www.sac-cas.ch/de/huetten-und-touren/sac-tourenportal/}{www.sac-cas.ch}, SAC-CAS App}
    \end{itemize}
  }
  \item{\textbf{Mensch} -- Gruppe, Technik, Kondition, Material}
\end{itemize}

Zu unterschiedlichen \textit{Zeitpunkten}:

\begin{itemize}
  \item{\textbf{Planung} -- zu Hause, in der Hütte}
  \item{\textbf{Vor Ort} -- Ausgangspunkt, Unterwegs}
  \item{\textbf{Einzelhang} -- vor jedem Hang, Schlüsselstelle}
\end{itemize}

Jeder Faktor muss zu jedem Zeitpunkt stimmig sein.
Sobald ein Faktor nicht mehr passt entsteht ein \textit{Riskio}.
Dieses Risiko muss dann mit einer geeigneten Massnahme vermindert werden.
\textbf{Bei zwei und mehr Risikofaktoren sollten wir verzichten oder die Tour abbrechen.}

\textit{Eine gefährliche Situation entsteht meist nicht zufällig und aus dem nichts, sondern ist das Resultat einer Verkettung unvorteilhafter Entscheidungen.}

\newcolumn

\subsection{Wie finde ich eine geeignete Tour?}

Die Auswahl einer geeigneten Tour ist nicht immer ganz einfach.
An erster Stelle steht natürlich immer die Sicherheit.
Für unsere Entscheidung können wir unter anderem folgende Kriterien einfliessen lassen:

\begin{itemize}
  \item{Welche Lawinensituation herrscht in dem Gebiet?}
  \item{
    Welches Wetter herrscht(e) in dem Gebiet?
    \begin{itemize}
      \item{Windstärke? Windrichtung?}
      \item{Sonnenschein oder Wolkendecke?}
      \item{Wie viel Niederschlag ist gefallen?}
      \item{Kalte oder eher warme Temperaturen?}
    \end{itemize}
  }
  \item{
    Wie ist die Schneedecke aufgebaut?
    \begin{itemize}
      \item{Liegt überhaupt genug Schnee?}
      \item{Pulverschnee oder Bruchharsch?}
    \end{itemize}
  }
  \item{Mit welcher Schwierigkeit ist die Tour bewertet?}
  \item{Höhenmeter, Distanz und Zeitbedarf?}
  \item{Ist die Tour stark oder schwach frequentiert?}
  \item{Erreichbarkeit mit dem ÖV oder Parkplätze?}
\end{itemize}

\subsection{Wo finde ich den besten Pulverschnee?}

Oft suchen wir auf Skitouren den besten Pulverschnee.
Um die Schneelage einzuschätzen eignet sich die \textit{Schneehöhenkarte} des SLF (wird jeweils um 7 Uhr aktualisiert).
Um die Neuschnee reichen Gebiete zu finden kann die \textit{Neuschneekarte} konsultiert werden.
Auch Webcams aus dem Gebiet können hilfreich sein.
Ausserdem können folgende Faustregeln aus der Meteorologie helfen:

\begin{itemize}
  \item{Pro 100 hm nimmt die Temperatur etwa 1 °C ab.}
  \item{Pro 1 mm Niederschlag fällt rund 1 cm Schnee.}
  \item{Schnee fällt ab der Nullgradgrenze.}
\end{itemize}

\textit{So können wir -- ausgehend von einer beliebigen Wetterstation im Gebiet -- die Chance auf Neuschnee in etwa abschätzen.}

\newcolumn

\subsection{Welche Touren eignen sich für Neulinge?}

Auch als absolute Skitouren Neulinge gibt es einige Touren, welche sich ohne grosse Erfahrung sicher begehen lassen:

\begin{itemize}
  \item{Laucherenstöckli, bei Ibergeregg}
  \item{Wildspitz, oberhalb Steinerberg}
  \item{Rägeflüeli, im Eigental beim Pilatus}
  \item{Rossstock, in der Lidernen}
  \item{Firsthöreli und Mattner First, im Bisistal}
  \item{Rickhubel, am Glaubenbergpass}
  \item{Mariannenhubel, im Diemtigtal}
  \item{Meniggrat, im Diemtigtal}
\end{itemize}

\subsection{Empfehlungen zur Sicherheit}

Der Sicherheit wegen noch ein paar Empfehlungen:

\begin{itemize}
  \item{Nur bei Lawinenstufe Mässig oder Gering gehen.}
  \item{Nur bei schönem Wetter und guter Sicht unterwegs sein -- der Helikopter fliegt bei Nebel nicht.}
  \item{Nicht alleine auf Skitour gehen.}
  \item{Bei Neuschnee etwas später aufbrechen, so ist ziemlich sicher bereits eine Spur vorhanden.}
  \item{Entlang vorhandener Spuren abfahren.}
  \item{Nichts überstürzen und genug Zeit einplanen.}
  \item{Im Zweifelsfall abbrechen.}
\end{itemize}

Zum Training eignet sich auch Aufstieg und Abfahrt in einem Skigebiet.
\textit{Dies wird in einigen Skigebieten allerdings nicht gerne gesehen.}
