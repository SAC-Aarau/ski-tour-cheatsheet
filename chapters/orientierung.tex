\section{Orientierung und Karten lesen}

\subsection{Welche Herausforderungen erwarten uns?}

Die Orientierung im winterlichen Gebirge hat seine ganz eigenen Herausforderungen.
Wir können nicht einfach einem Pfad, Wegspuren, Steinmännern oder Markierungen folgen.
Sondern müssen uns -- falls keine Spur vorhanden ist -- den Weg selbst suchen.
Auch eine vorhandene Spur ist mit Vorsicht zu geniessen und nicht immer hilfreich.

\subsection{Was kann uns bei der Orientierung unterstützen?}

Einige Hilfsmittel können uns hier unterstützen:

\begin{itemize}
  \item{Höhenmesser, Kompass, Papierkarten}
  \item{GPS, Smartphone}
  \item{\href{https://s.geo.admin.ch/y34y7btkqsvz}{www.map.geo.admin.ch}}
  \item{\href{https://www.skitourenguru.com//}{www.skitourenguru.com}}
  \item{White Risk App (kostenpflichtig)}
  \item{swisstopo App}
\end{itemize}

\textit{Smartphone und GPS sind nützlich -- wir sollten uns aber niemals ausschliesslich darauf verlassen.}

\subsection{Karten in Papierform?}

Eine Karte in Papierform kann nicht kaputt gehen, hat keine Batterie oder Akku und ist somit ein sehr robustes Navigationsmittel.

\textit{Vorsicht bei selbst gedruckten Karten, ihr Massstab kann täuschen.}

\subsection{Was sind die Hangneigungsklassen?}

Im Winter besonders hilfreich sind die sogenannten Hangneigungsklassen.
Hier ist die Hangneigung ab 30° in 5° Schritten farblich kodiert:

\cbox[white]{10pt} keine Färbung, mässig steil, unter 30°\\
\cbox[above30]{10pt} gelb, steil, über 30°\\
\cbox[above35]{10pt} orange, sehr steil, über 35°\\
\cbox[above40]{10pt} rot, extrem steil, über 40°\\
\cbox[tooSteep]{10pt} violett, über 45°

Im Gelände können wir folgende Faustregeln verwenden um die Hangneigung abzuschätzen:

\begin{itemize}
  \item{Unter 30° können wir Kurven laufen.}
  \item{Ab 30° müssen wir Spitzkehren machen.}
  \item{Fels durchsetztes Gelände ist oft über 40° steil.}
\end{itemize}

\newcolumn

\subsection{Was kann die Orientierung erschweren?}

Schlechte Sicht durch diffuse Verhältnisse, Nebel oder Schneefall können die Orientierung im Gelände erheblich erschweren.
Im Zweifelsfall können wir der vorhandenen Spur entlang zurückfinden.
Vorsicht aber bei starkem Wind oder Schneefall, dieser kann uns in kurzer Zeit die Spur zunichte machen.

\subsection{Schutzgebiete und Wildruhezonen}

Nicht ausser Acht lassen sollten wir Wildruhezonen und Schutzgebiete.
Wir unterscheiden zwei Kategorien:

\begin{itemize}
  \item{\textbf{Rechtsverbindlich} -- hier droht ein Bussgeld}
  \item{\textbf{Empfohlen} -- nach Möglichkeiten meiden}
\end{itemize}

In der Regel finden sich am Ausgangspunkt und manchmal entlang der Route im Gelände Hinweistafeln.
Offizielle  Routen und Korridore sind meist auch im Winter mit Wegweiser gekennzeichnet.
\href{https://www.respektiere-deine-grenzen.ch/}{www.respektiere-deine-grenzen.ch}
